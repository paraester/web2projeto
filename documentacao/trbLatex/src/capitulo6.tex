\chapter{CONFIGURAÇÕES}
\section{Classe Screen}

\begin{lstlisting}
import java.applet.Applet;
import java.awt.*;
import java.awt.event.*;
import javax.swing.JOptionPane;
public class Screen extends Applet implements ActionListener {
    private Button botaoAdicionaRainha;
    private Button botaoRemoveRainha;
    private Button botaoLimpar;
	private TabuleiroGrafico tabuleiro;    
    private TextField coordenadaLinha;
    private TextField coordenadaColuna;
    private String colunas[] = {"-", "a","b","c","d","e","f","g","h"};
    private Graphics gr;
    public void Screen() {
        init();
    }
    @Override 
    public void init() {
        gr = this.getGraphics();
        botaoAdicionaRainha = new Button("adiciona");
        botaoRemoveRainha = new Button("remove");
        botaoLimpar = new Button("Limpar");
        tabuleiro = new TabuleiroGrafico();
        coordenadaLinha = new TextField(5);
        coordenadaColuna = new TextField(5);  
        setLayout(new BorderLayout());
        setSize(800, 600);
        setBackground(Color.white);
        add("Center", tabuleiro);   
        Panel bPanel = new Panel();
        add("South", bPanel);  
        bPanel.setLayout(new GridLayout(1,6));
        bPanel.add(new Label(" Coluna de a - h"));
        bPanel.add(coordenadaLinha);    
        bPanel.add(new Label(" Linha de 1-8"));
        bPanel.add(coordenadaColuna);   
        bPanel.add(botaoAdicionaRainha);
        bPanel.add(botaoRemoveRainha);
        bPanel.add(new Label(" "));
        bPanel.add(botaoLimpar);
        bPanel.setBackground(Color.BLACK);
        bPanel.setForeground(Color.YELLOW);
        botaoRemoveRainha.addActionListener(this);
        botaoAdicionaRainha.addActionListener(this);
        botaoLimpar.addActionListener(this);
        JOptionPane.showMessageDialog(null, "Oito  Rainhas desafio logico de dispor oito peças em um tabuleiro de forma que nenhuma seja atacada por outra. \n + Assim, faz-se necessário que duas rainhas quaisquer não estejam numa mesma linha, coluna ou diagonal.\n");        
    }
    @Override
    public void actionPerformed(ActionEvent e) { 
        String linha = coordenadaLinha.getText();
        String coluna = coordenadaColuna.getText();
        coordenadaLinha.setText("");
        coordenadaColuna.setText(""); 
        int x = coordX(linha);
        int y = coordY(coluna);
        if (e.getSource() == botaoLimpar) { 
            showDialog(tabuleiro.botaoLimpar());
            
        } else if (e.getSource() == botaoAdicionaRainha) {             
            showDialog(tabuleiro.botaoAdiciona(x, y));

        } else if (e.getSource() == botaoRemoveRainha) {                   
            showDialog(tabuleiro.botaoRemove(x, y));
        }
    }
    private void showDialog(Mensagem mensagens) {
        if (mensagens.getText().equals("ok"))
        {
            return;
        }
        JOptionPane.showMessageDialog(null, mensagens.getText());    
    }
    public int coordX(String linha){
        switch( linha )
        {
            case "a": return 0;
            case "b": return 1;
            case "c": return 2;
            case "d": return 3;
            case "e": return 4;
            case "f": return 5;
            case "g": return 6;
            case "h": return 7;
            default: return -1;
        }    

    }
      public int coordY(String coluna){
        switch( coluna )
        {
            case "8": return 0;
            case "7": return 1;
            case "6": return 2;
            case "5": return 3;
            case "4": return 4;
            case "3": return 5;
            case "2": return 6;
            case "1": return 7;
            default: return -1;
        } 
    }
}
\end{lstlisting}
\newpage
\section{Classe TabuleiroGrafico}

\begin{lstlisting}
import java.awt.*;
public class TabuleiroGrafico extends Canvas 
{
    private Graphics parteGrafica;
    private Tabuleiro tabuleiroLogico;
    private int larguracasinha;
    private int alturacasinha; 
    public TabuleiroGrafico()
    {
        this.tabuleiroLogico = new Tabuleiro();
        this.parteGrafica = this.getGraphics();
    }
    @Override
    public void paint(Graphics g) {
        pintarTabuleiro(g);
    }   
    public void pintarTabuleiro(Graphics g) {
        parteGrafica = g;
        larguracasinha = getBounds().width/10;
        alturacasinha = getBounds().height/10;
        String colunas[] = {"-", "a","b","c","d","e","f","g","h"};
        String linhas[] = {"-","8","7","6","5","4","3","2","1"};
        for (int i = 0; i < 8; i++) {
            for (int j = 0; j < 8; j++) {
                desenhaCasinha(i,j);
            }
        }
        parteGrafica.setColor(Color.black);
        int deslocar = alturacasinha/4;
        for (int i = 1; i < 9; i++) { 
            parteGrafica.drawString(colunas[i], i*larguracasinha + larguracasinha/2, alturacasinha/2 + deslocar);
            parteGrafica.drawString(colunas[i], i*larguracasinha + larguracasinha/2, alturacasinha/2  + alturacasinha * 9);
            parteGrafica.drawString(linhas[i], larguracasinha/2 + deslocar, i*alturacasinha + alturacasinha/2 );
            parteGrafica.drawString(linhas[i], larguracasinha/2 -deslocar*2 + larguracasinha * 9, i*alturacasinha + alturacasinha/2 );
        }
    }
    private void desenhaCasinha(int x, int y) {
        int coresAlternadas = 0;
        parteGrafica.setColor(Color.yellow);
        parteGrafica.drawRect((x+1)*larguracasinha, (y+1)*alturacasinha, larguracasinha, alturacasinha); //(y+1) para transladar
        if ((x + y) % 2 == 0) {    
            coresAlternadas = 10;
        }     
        switch (tabuleiroLogico.getValor(x, y)) {
            case 0:
                parteGrafica.setColor(new Color(230+coresAlternadas, 230+coresAlternadas, 230+coresAlternadas)); 
                break;
            case 1:
                parteGrafica.setColor(new Color(230+coresAlternadas, 230+coresAlternadas, 230+coresAlternadas)); 
              //parteGrafica.setColor(new Color(160,110,110)); //se quiser ver as dicas
                break;
            case 2:
                parteGrafica.setColor(new Color(0,0,0));
                break;
        }
        parteGrafica.fillRect((x+1)*larguracasinha+1, (y+1)*alturacasinha+1, larguracasinha-1, alturacasinha-1);
        insereCoroa(x, y);
    }
    private void insereCoroa(int x, int y)
    {
        if (tabuleiroLogico.getValor(x, y) != 2) {
            return;
        }
        Dimension xy = getSize(); //Dimensoes padrão são 800 por 569 pra exibir coroa
        parteGrafica.setColor(Color.red);
        if (xy.width < 569 || xy.height < 569) {
            int coordx = (x+1)*larguracasinha + (larguracasinha-6)/2;
            int coordy = (y+1)*alturacasinha + (alturacasinha+6)/2;
            parteGrafica.drawString("W", coordx, coordy);                                            
            return;
        }
        int mediaa = (larguracasinha - 50)/2; 
        int mediab = (alturacasinha - 50)/2;
        int a = (x+1)*larguracasinha + mediaa;
        int b = (y+1)*alturacasinha + mediab;
        parteGrafica.drawLine(a+5, b+45, a+45, b+45);
        parteGrafica.drawLine(a+5, b+45, a+5, b+5);
        parteGrafica.drawLine(a+45, b+45, a+45, b+5);
        parteGrafica.drawLine(a+5, b+35, a+45, b+35);
        parteGrafica.drawLine(a+5, b+5, a+15, b+20);
        parteGrafica.drawLine(a+15, b+20, a+25, b+5);
        parteGrafica.drawLine(a+25, b+5, a+35, b+20);
        parteGrafica.drawLine(a+35, b+20, a+45, b+5);
        parteGrafica.drawOval(a+20, b+20, 10, 10);
    }
    public Mensagem botaoLimpar() { 
        parteGrafica = this.getGraphics();
        tabuleiroLogico.iniciarTabuleiro();
        pintarTabuleiro(parteGrafica);
        return new Mensagem(0,0,0); 
    }
    public Mensagem botaoRemove(int x, int y) {
        parteGrafica = this.getGraphics();
        if (x < 0 || y < 0) {
            return new Mensagem(99,0,0);  
        }
        if (tabuleiroLogico.temRainha(x, y)) {
            tabuleiroLogico.removeRainha(x, y);
            pintarTabuleiro(parteGrafica);
        } else {
            return new Mensagem(5,x,y);
        }            
        return new Mensagem(0,0,0);
    }
    public Mensagem botaoAdiciona(int x, int y) {
        parteGrafica = this.getGraphics();
        if (x < 0 || y < 0) {
            return new Mensagem(99,0,0); 
        }   
        if (tabuleiroLogico.aceitaRainha(x, y)){
            tabuleiroLogico.adicionaRainha(x, y);         
            pintarTabuleiro(parteGrafica);
            if ( tabuleiroLogico.fimDeJogo() ) {
                return new Mensagem(100,0,0);
            }
        } else {
           return tabuleiroLogico.rainhaAtaca(x,y);
        }
        return new Mensagem(0,0,0);
    }
}

\end{lstlisting}
\newpage
\section{Classe Tabuleiro}

\begin{lstlisting}
public class Tabuleiro {
    private int larguracasinha;
    private int alturacasinha;
    private int quantidadeRainhas = 0;
    private int[][] casasTabuleiro = new int[8][8];
    private String colunas[] = {"-", "a","b","c","d","e","f","g","h"};
    public Tabuleiro() {
        iniciarTabuleiro();   
        imprimeTabuleiro();
    }
    public void iniciarTabuleiro(){
        quantidadeRainhas = 0;
        for (int i = 0; i < 8; i++) {
            for (int j = 0; j < 8; j++) {
                casasTabuleiro[i][j] = 0;
            }
        }
    }
    public int getValor(int x, int y) {
        if (dentroTabuleiro(x,y)) {
            return casasTabuleiro[x][y];
        }
        return -1;
    }
    public boolean temRainha(int x, int y)
    {
        if (dentroTabuleiro(x,y) && casasTabuleiro[x][y] == 2)
        {
            return true;           
        }
        return false;
    }
    public boolean aceitaRainha(int x, int y)
    {
        if (dentroTabuleiro(x,y) && casasTabuleiro[x][y] == 0)
        {
           return true;
        }
       return false;
    }
    private boolean dentroTabuleiro(int x, int y) {
        return (x >= 0 && x < 8 && y >= 0 && y < 8);
    }
    public void adicionaRainha(int x, int y) {
        quantidadeRainhas++;
        casasTabuleiro[x][y] = 2;
        administraTabuleiro(x,y,true);     
        imprimeTabuleiro();
    }
    public void removeRainha(int x, int y) {
        casasTabuleiro[x][y] = 0;
        administraTabuleiro(x,y,false);
        quantidadeRainhas = 0; 
        for (int i = 0; i < 8; i++) {
            for (int j = 0; j < 8; j++) {
                if (casasTabuleiro[i][j] == 2) {
                    adicionaRainha(i,j);
                }
            }
        }
        imprimeTabuleiro();
    }
    public void administraTabuleiro(int x, int y, boolean adicionaRemove){
        int i, valor;
        if (adicionaRemove){
            valor = 1;
        } else {
            valor = 0;
        }
        for (i = 0; i < 8; i++) {
            if (i != y && casasTabuleiro[x][i] != 2) {
                casasTabuleiro[x][i] = valor;
            }
            if (i != x && casasTabuleiro[i][y] != 2) {    
            casasTabuleiro[i][y] = valor;
            }
        }
        for (i=1; dentroTabuleiro(x-i,y+i); i++) {
                casasTabuleiro[x-i][y+i] = valor;
        }
        for (i=1; dentroTabuleiro(x-i,y-i); i++) {
            casasTabuleiro[x-i][y-i] = valor;
        }
        for (i=1; dentroTabuleiro(x+i,y+i); i++) {
            casasTabuleiro[x+i][y+i] = valor;
        }
        for (i=1; dentroTabuleiro(x+i,y-i); i++) {
            casasTabuleiro[x+i][y-i] = valor;
        }
    }
    public Mensagem rainhaAtaca(int x, int y){
        int limitesupi = x;
        int limitesupj = y;
        int i = x; 
        int j = y; 
        if (dentroTabuleiro(x,y) && casasTabuleiro[x][y] == 2)
        {
            return new Mensagem(1, x, y);
        }
        for (int d = 1; d < 8; d++) {
            i--;
            j--;
            limitesupi++;
            limitesupj++;
            for (int z = i; z <= limitesupi; z++) {
                for (int w = j; w <= limitesupj; w++) {
                    if (dentroTabuleiro(z,w) && casasTabuleiro[z][w] == 2 && constataAtaque(x,y,z,w)){
                        return new Mensagem(2, z, w);
                    }
                }
            }
        }
        return new Mensagem(2, x, y);
    }
    private boolean constataAtaque(int x, int y, int a, int b)
    { 
      if (x == a || y == b )  {
          return true;
      }
      if ((x-a) == (y - b)){ 
          return true;
      }
      if ((-1)*(x-a) == (y -b)){ 
          return true; 
      }
      return false;
    }
    public boolean fimDeJogo(){
        return quantidadeRainhas == 8;
    }
    private void imprimeTabuleiro()
    {
        //System.out.println("Agora estamos com " + quantidadeRainhas + " rainha(s) no tabuleiro");
        for (int i = 0; i < 8; i++) {
            for (int j = 0; j < 8; j++) {
                //System.out.print(casasTabuleiro[i][j] + " ");
            }
            System.out.println(" ");
        }
    }   
}

\end{lstlisting}
\newpage
\section{Classe Mensagem}

\begin{lstlisting}
public class Mensagem {
    private final int erro;
    private final int coordx;
    private final int coordy;   
    public Mensagem(int e, int x, int y){
        this.coordx = x;
        this.coordy = y;
        this.erro = e;  
    }
    private int getErro(){
        return this.erro;
    }
    private int getX(){
        return this.coordx;
    }
    private int getY(){
        return this.coordy;
    }
    public String getText() {
        int e, x, y;
        e = getErro();
        x = getX();
        y = getY();
        String colunas[] = {"-", "a","b","c","d","e","f","g","h"};
        switch( e ) {
            case 0: 
              return "ok";
            case 1:
                return "Ja existe uma Rainha na coluna '" + colunas[x+1] + "' e na linha '" +  (8-y) +"'!";                  
            case 2:
                return "Existe conflito! Sofre ataque da Rainha da coluna '" + colunas[x+1] + "' e da linha '" +  (8-y) +"'";                  
            case 3: 
                return "Existe conflito na linha '" + colunas[x+1]  + "' e coluna '" + (8-y) + "' mas nao achamos com qual rainha ";                  
            case 5: 
                return "Nao existe rainha na coluna '" + colunas[x+1] + "' e linha '" + (8-y) + "'";                  
            case 99: 
                return "Valores invalidos para coluna ou linha!"; 
            case 100: 
               return "Parabens Voce conseguiu colocar as 8 rainhas!!"; 
            default:
               return "Valores invalidos!";
        }         
    }   
}
\end{lstlisting}