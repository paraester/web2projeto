\chapter{Exigências Específicas}
	O sistema deve ser implementado contendo as funcionalidades referentes aos tópicos apresentados abaixo. Cada tópico será avaliado separadamente. Para o aluno obter nota máxima em cada tópico, ele precisa utilizar todas as estruturas listadas nos respectivos sub-tópicos. Também serão consideradas as boas práticas de programação em JavaScript, uso adequado de notações e conceitos aprendidos, organização do código e criatividade.

\section{Qualidade do código}


\subsection{Style Guide}
	Objetivo: apresentar o uso de apenas 10 regras do style escolhido.

\subsection{Lint}
	Objetivo: mostrar a correção de apenas 5 problemas informados pelo lint.

\subsection{Strict mode}
	


\section{Caixas de Diálogo}

\subsection{prompt}

\subsection{alert}

	No arquivo $JavaScript$ chamado $contato.js$ criamos uma função com nome $validaContato()$ conforme segue abaixo para melhor visualização:
\begin{lstlisting}
function validaContato() {
    var nome = $$("name").value;
    console.log("nome" + nome);
    var email = $$("email").value;
    var message = $$("message").value;
    if ((nome == null) || (nome == "")) {
        alert("Preencha o campo nome");
        return false;
    } else if ((email == null) || (email == "")) {
        alert("Preencha o campo email");
        return false;
    } else if ((message == null) || (message == "")) {
        alert("Preencha o campo mensagem");
        return false;
    }
    return true;
}
\end{lstlisting}
	Esta função está relacionada a validação dos campos do formulário. E dentro dela utilizamos $alert$ para enviar uma mensagem ao visitante da página.

\subsection{confirm}

\section{Funções}
\subsection{Função anônima com argumento}
\subsection{Função anônima sem argumento}
\subsection{Função anônima auto-executável}
\subsection{Função com nome}
	No arquivo $JavaScript$ chamado $contato.js$ criamos uma função com nome $validaContato()$ conforme segue abaixo para melhor visualização:
\begin{lstlisting}
function validaContato() {
    var nome = $$("name").value;
    console.log("nome" + nome);
    var email = $$("email").value;
    var message = $$("message").value;
    if ((nome == null) || (nome == "")) {
        alert("Preencha o campo nome");
        return false;
    } else if ((email == null) || (email == "")) {
        alert("Preencha o campo email");
        return false;
    } else if ((message == null) || (message == "")) {
        alert("Preencha o campo mensagem");
        return false;
    }
    return true;
}
\end{lstlisting}
	Esta função está relacionada a validação dos campos do formulário.

\subsection{Função aninhada/interna}
\subsection{Passagem de uma função como parâmetro}

\section{Eventos}
\subsection{Evento de carregamento do documento}


\subsection{Evento de movimento do mouse}


\subsection{Evento de teclado}
 Objetivo:  - usar charCode ou KeyCode.
 
 
\subsection{Eventos de formulário}
Objetivo: onfocus e onblur.


\subsection{objeto event}
Obejtivo: Imprimir alguma propriedade do objeto event recebido como parâmetro

\subsection{Propagação de eventos no modelo bolha}

\section{Acesso aos elementos DOM do HTML }
\subsection{Via acesso direto}
 Pelo id do elemento HTML
\subsection{Via getElementByID()}


\subsection{Via getElementsByName()}



\subsection{Via getElementsByTagName()}



\subsection{Via seletores CSS}
  Usados na função querySelector() ou jQuery
  
\section{Tratadores de Evento}
\subsection{Evento inline}
Objetivo: especificar o tratador de evento inline

tratador de eventos inline - onmouseover, onmouseout, onclick



\subsection{Modo tradicional}
Objetivo: especificar o tratador de evento no carregamento da página HTML no modo tradicional.
\subsection{addEventListener}
 Objetivo: especificar o tratador de evento no carregamento da página HTML com a função addEventListener.
 
\subsection{Operador this}
 Objetivo: usar o operador this em funções tratadoras de eventos.
\subsection{•}


\section{Formulário}
	Foi criado a página $contato.html$. Nesta página temos um formulário com $3$ campos de preenchimento obrigatórios.
	
\subsection{Validação}
	Objetivo: validação de formulário com onsubmit usando os métodos tradicionais.
	
	Para cumprir o objetivo o formulário $contato.html$ faz uso da seguinte maneira:
\begin{lstlisting}
<form id="contact-form" class="validate-form" method="post" onsubmit="return validaContato()">
\end{lstlisting}

	O arquivo $JavaScript$ chamado $contato.js$ contém a função com nome $validaContato()$:
\begin{lstlisting}
function validaContato() {
    var nome = $$("name").value;
    console.log("nome" + nome);
    var email = $$("email").value;
    var message = $$("message").value;
    if ((nome == null) || (nome == "")) {
        alert("Preencha o campo nome");
        return false;
    } else if ((email == null) || (email == "")) {
        alert("Preencha o campo email");

        return false;
    } else if ((message == null) || (message == "")) {
        alert("Preencha o campo mensagem");

        return false;
    }
    enviadoContato();
    return true;
}
\end{lstlisting}
	

\subsection{Propriedade Value}
	Objetivo: ler e escrever em elementos $input$ com a propriedade $value$
\begin{lstlisting}
$(document).ready(function () {
    $('input').val('Informacoes enviadas')
})
\end{lstlisting}


\subsection{innerHtml}
	Objetivo: alterar o conteúdo de elementos $div$ ou $p$com a propriedade $innerHTML$
	
	Para este item escolhemos alterar o elemento $div$ do formulário de $contatos$ pois assim que o visitante enviar uma mensagem de contato a estrutura do formulário será substituída por uma frase, conforme código abaixo, definido pela função.

\begin{lstlisting}
function enviadoContato() {
    document.getElementById("contact-form").innerHTML = ("REGISTROS ENVIADOS");
}
\end{lstlisting}


\subsection{checkbox, radio ou select}
Manipulação de elemento de listagem, como checkbox, radio ou select


\subsection{Acesso via hierarquia}
Acesso aos elementos de um formulário via hierarquia (caminho) de objetos, ou seja, array forms e elements



\section{Objetos Nativos }
\subsection{Manipulação de array}
Usar métodos para manipulação de array
\subsection{Manipulação de string}
Usar métodos para manipulação de string

\section{Objetos Nativos}
\subsection{Manipulação de array}
Usar métodos para manipulação de array

jsonClienteX na funcao alternarEscolhaDaFoto no arquivo cliente.html



\subsection{Manipulação de string}


Usar métodos para manipulação de string


\section{Objetos}
\subsection{Criar objeto}
Criar objeto usando função construtora ou notação literal


\subsection{Herança}
Usar herança prototipal


\section{jQuery}
\subsection{Seletores CSS}
Uso de seletores CSS - id, classe e tag
	
	ver página cliente.html


\subsection{Seletores hierarquicos}
Uso de seletores hierarquicos - ancestral/descendente, pai/filho, anterior/proximo
\subsection{Efeitos fade ou slide}

Efeito fadeOut ver cliente.html


\subsection{Tratador de algum evento}
Espeficar o tratador de algum evento via jQuery
\subsection{Manipulação CSS}
Manipulação do CSS via função css() e addClass()/remoceClass()

Manipulação do CSS addClass removeClasse ver function alternarEscolhaDaFoto(i) em cliente.html


\subsection{Manipulação do conteúdo}
Manipulação do conteúdo de um input e div usando jQuery

\section{Web Storage }
\subsection{LocalStorage e SessionStorage}
\subsection{Leitura e escrita de dados simples}
\subsection{Leitura e escrita de JSON}
