\chapter{Exigências Específicas}
	O sistema deve ser implementado contendo as funcionalidades referentes aos tópicos apresentados abaixo. Cada tópico será avaliado separadamente. Para o aluno obter nota máxima em cada tópico, ele precisa utilizar todas as estruturas listadas nos respectivos sub-tópicos. Também serão consideradas as boas práticas de programação em JavaScript, uso adequado de notações e conceitos aprendidos, organização do código e criatividade.

\section{Qualidade do código}


\subsection{Style Guide}
	Objetivo: apresentar o uso de apenas 10 regras do style escolhido.

\subsection{Lint}
	Objetivo: mostrar a correção de apenas 5 problemas informados pelo lint.

\subsection{Strict mode}
	


\section{Caixas de Diálogo}

\subsection{prompt}

\subsection{alert}

\subsection{confirm}

\section{Funções}
\subsection{Função anônima com argumento}
\subsection{Função anônima sem argumento}
\subsection{Função anônima auto-executável}
\subsection{Função com nome}
\subsection{Função aninhada/interna}
\subsection{Passagem de uma função como parâmetro}

\section{Eventos}
\subsection{Evento de carregamento do documento}
\subsection{Evento de movimento do mouse}
\subsection{Evento de teclado}
 Objetivo:  - usar charCode ou KeyCode.
\subsection{Eventos de formulário}
Objetivo: onfocus e onblur.
\subsection{objeto event}
Obejtivo: Imprimir alguma propriedade do objeto event recebido como parâmetro

\subsection{Propagação de eventos no modelo bolha}

\section{Acesso aos elementos DOM do HTML }
\subsection{Via acesso direto}
 Pelo id do elemento HTML
\subsection{Via getElementByID()}
\subsection{Via getElementsByName()}
\subsection{Via getElementsByTagName()}
\subsection{Via seletores CSS}
  Usados na função querySelector() ou jQuery
  
\section{Tratadores de Evento}
\subsection{Evento inline}
Objetivo: especificar o tratador de evento inline
\subsection{Modo tradicional}
Objetivo: especificar o tratador de evento no carregamento da página HTML no modo tradicional.
\subsection{addEventListener}
 Objetivo: especificar o tratador de evento no carregamento da página HTML com a função addEventListener.
 
\subsection{Operador this}
 Objetivo: usar o operador this em funções tratadoras de eventos.
\subsection{•}


\section{Formulário}
\subsection{Validação}
Validação de formulário com onsubmit usando os métodos tradicionais
\subsection{Propriedade Value}
Ler e escrever em elementos input com a propriedade value
\subsection{innerHtml}
Alterar o conteúdo de elementos div ou p com a propriedade innerHTML
\subsection{checkbox, radio ou select}
Manipulação de elemento de listagem, como checkbox, radio ou select
\subsection{Acesso via hierarquia}
Acesso aos elementos de um formulário via hierarquia (caminho) de objetos, ou seja, array forms e elements

\section{Objetos Nativos }
\subsection{Manipulação de array}
Usar métodos para manipulação de array
\subsection{Manipulação de string}
Usar métodos para manipulação de string

\section{Objetos}
\subsection{Criar objeto}
Criar objeto usando função construtora ou notação literal
\subsection{Herança}
Usar herança prototipal


\section{jQuery}
\subsection{Seletores CSS}
Uso de seletores CSS - id, classe e tag
\subsection{Seletores hierarquicos}
Uso de seletores hierarquicos - ancestral/descendente, pai/filho, anterior/proximo
\subsection{Efeitos fade ou slide}
\subsection{Tratador de algum evento}
Espeficar o tratador de algum evento via jQuery
\subsection{Manipulação CSS}
Manipulação do CSS via função css() e addClass()/remoceClass()
\subsection{Manipulação do conteúdo}
Manipulação do conteúdo de um input e div usando jQuery

\section{Web Storage }
\subsection{LocalStorage e SessionStorage}
\subsection{Leitura e escrita de dados simples}
\subsection{Leitura e escrita de JSON}
