%Exemplo de resumo

Resumo.

O Problema das Oito Rainhas consiste no desafio lógico de dispor oito peças (rainha) em um tabuleiro de xadrez (tradicional, $8x8$) de forma que nenhuma peça seja atacada por outra. Assim, faz-se necessário que duas rainhas quaisquer não estejam numa mesma linha, coluna ou diagonal. O problema possui 92 soluções distintas que na realidade tratam-se de 12 soluções originais de onde obtemos as outras soluções por operações de simetria (rotação e reflexão). O presente projeto apresentará um aplicativo que gerencia a colocação das oito damas/rainhas, uma por vez, no tabuleiro de xadrez. O jogador poderá inserir e retirar cada rainha deste tabuleiro mediante coordenadas (que vão de “$a$” até “$h$” e de $1$ até $8$, conforme o padrão internacional). A cada movimento, constatará e informará ao jogador onde ocorreu o primeiro ataque. Sinalizaremos quando o problema for resolvido. Frisamos que este aplicativo não irá catalogar as 92 possíveis soluções, mas verificar se o estágio atual de resolução condiz com as restrições do problema proposto.